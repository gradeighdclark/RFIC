\section{Circuit Design}
include three sub-sections: one for LNA, one for Mixer and one for VCO. For each subsection discuss the design procedure (describe the circuit architecture, provide justification, and describe your approach for determining the size of transistors and the values of passive elements), include transistor-level schematic, simulation results, and a table summarizing the performance metrics of the circuit.

\subsection{Low Noise Amplifier}

\subsection{Mixer}

\subsection{Voltage Controlled Oscillator}
As shown in Figure /ref{vco}, the proposed single-ended VCO uses a pair of complementary np-MOSFETs so that the dc current can be reused and a low power VCO can be realized [1]. The LC tank will determine the oscillation frequency by:

f_0 = 1 / 2pi sqrt(L_2 C_v)

Note that, Cv includes varactors, gate-source and drain-source capacitors of np-MOSFETs. The varactors are used to tune the VCO oscillation frequency. And on the other hand, the transistors are configured to provide a negative resistance to compensate the tank loss. 

FIGURE HERE

The nMOS has the dimension of $L/W=0.18/30$ and the pMOS has the dimension of $L/W=0.18/200$, both in m. The values for inductor and varactors are optimized to center the oscillation frequency at 400 MHz. Note that, for better estimation of the VCO performance, all the capacitors, inductors and resistors are used as non-ideal components in the Cadence simulation. 

