\section{Introduction}
The accelerating growth of both aging populations and the number of technologically savvy, health minded citizens around the world comes with them the requirements to provide advanced new medical solutions to meet growing demand. Some of the ways these solutions have presented themselves over the last five to six years were in the form of new designs involving both integrated circuit and medical technology devices. The devices could be implantable (e.g. cochlear implants, pacemakers) or body worn (health monitoring devices, e.g. accelerometer). 

To implement these technologies requires reliable wireless links that need to transmit over several feet. To meet that end, the FCC announced in 2009 the establishment of a dedicated amount of spectrum to wireless medical device technology, entitled MedRadio~\cite{fcc}. Selected for a frequency band that would not cause interference with other devices in a patient's daily life, MedRadio occupies the spectrum from 401 to 406 MHz and is subdivided into the following sub-bands~\cite{cha1, fcc}:
\begin{itemize}
	\item Two ``wing'' bands between 401-402 MHz and 405-406 MHz. Wing bands can be used for either body-worn devices or fully implantable ones.
	\item One ``core'' band between 402-405 MHz. The core band can be used \emph{only} for fully implanted devices.
\end{itemize}

Additionally, the devices need to consume relatively low power while possessing the smallest possible physical footprint~\cite{Bradley}. Reducing the power requires either reducing the supply voltage (typical applications hover around 1V to 1.8V)~\cite{Bradley} or reducing the current required to drive the circuitry. Reducing the dimensions of the device necessitates smaller process technologies (typically 0.18 micron or lower) with minimizing circuit components as well as employing tight packaging skills when creating layouts for the circuit. The wireless nature requires that the range of the device be, nominally, greater than 2 meters~\cite{Bradley}. This requires that the device sensitivity be quite low (around -80 dBm and lower) while having good matching to reduce reflective losses in receiving and transmitting.

Designing circuits to meet the constraints can be quite challenging for a designer. Reducing power consumption is the most important factor, but that typically comes at a cost to other values that necessitate the proper operation of the circuit such as a high gain, minimum noise figure, and maximum sensitivity. Moving forward, we present our implementation of a MedRadio receiver that seeks to meet all of the design constraints and more.
